%%%%%%%%%%%%%%%%%%%%%%%%%%%%%%%%%%%%%%%%%%%%%%%%%%%%%%%%%%%%%%%%%%%%%
%
% CSCI 1430 Writeup Template
%
% This is a LaTeX document. LaTeX is a markup language for producing
% documents. Your task is to fill out this
% document, then to compile this into a PDF document.
%
% TO COMPILE:
% > pdflatex thisfile.tex
%
% For references to appear correctly instead of as '??', you must run
% pdflatex twice.
%
% If you do not have LaTeX and need a LaTeX distribution:
% - Departmental machines have one installed.
% - Personal laptops (all common OS): www.latex-project.org/get/
%
% If you need help with LaTeX, please come to office hours.
% Or, there is plenty of help online:
% https://en.wikibooks.org/wiki/LaTeX
%
% Good luck!
% James and the 1430 staff
%
%%%%%%%%%%%%%%%%%%%%%%%%%%%%%%%%%%%%%%%%%%%%%%%%%%%%%%%%%%%%%%%%%%%%%
%
% How to include two graphics on the same line:
%
% \includegraphics[\width=0.49\linewidth]{yourgraphic1.png}
% \includegraphics[\width=0.49\linewidth]{yourgraphic2.png}
%
% How to include equations:
%
% \begin{equation}
% y = mx+c
% \end{equation}
%
%%%%%%%%%%%%%%%%%%%%%%%%%%%%%%%%%%%%%%%%%%%%%%%%%%%%%%%%%%%%%%%%%%%%%%%%%%%%%%%%%%%%%%%%%%%%%%%%

\documentclass[11pt]{article}

\usepackage[english]{babel}
\usepackage[utf8]{inputenc}
\usepackage[colorlinks = true,
            linkcolor = blue,
            urlcolor  = blue]{hyperref}
\usepackage[a4paper,margin=1.5in]{geometry}
\usepackage{stackengine,graphicx}
\usepackage{fancyhdr}
\setlength{\headheight}{15pt}
\usepackage{microtype}
\usepackage{times}
\usepackage{booktabs}

% python code format: https://github.com/olivierverdier/python-latex-highlighting
\usepackage{pythonhighlight}

\frenchspacing
\setlength{\parindent}{0cm} % Default is 15pt.
\setlength{\parskip}{0.3cm plus1mm minus1mm}

\pagestyle{fancy}
\fancyhf{}
\lhead{Homework 2 Writeup}
\rhead{CSCI 1430}
\rfoot{\thepage}

\date{}

\title{\vspace{-1cm}Homework 2 Writeup}


\begin{document}
\maketitle
\vspace{-2cm}
\thispagestyle{fancy}

\section*{Instructions}
\begin{itemize}
  \item This write-up is intended to be `light'; its function is to help us grade your work.
  \item Please describe any interesting or non-standard decisions you made in writing your algorithm.
  \item Show your results and discuss any interesting findings.
  \item List any extra credit implementation and its results.
  \item Feel free to include code snippets, images, and equations.
  \item Use as many pages as you need, but err on the short side.
  \item \textbf{Please make this document anonymous.}
\end{itemize}

\section*{Assignment Overview}
In this project, we were responsible for creating a program that creates a local feature matching algorithm to match multiple views of real-world scenes. Here, we were told to implement a simplified version of SIFT. SIFT is an algorithm used to detect, describe, and match local features in images.

\section*{Implementation Detail}
This project asked us to implement the three major methods of local feature matching:
\begin{itemize}
    \item getfeaturepoints
    \begin{itemize}
        \item Here we were asked to implement the Harris corner detector algorithm. The Harris corner detector uses approximations and linear algebra to reduce the cost of calculating a corner score for each point in the image.
    \end{itemize}
    
    \item getfeaturedescriptors
    \begin{itemize}
        \item For this method we were asked to implement a SIFT-like local feature descriptor. This told us that we had to compute the similarities between images. According to the handout, features consist of an interest point and it tells us about a point's immediate neighbor.  
    \end{itemize}
    
    \item matchfeatures
    \begin{itemize}
        \item For the third implementation we were asked to do, we had to implement a nearest neighbor distance ratio. This algorithm, as described in its name, returns the distance between two features. With this, we would be able to match different features to one another.
    \end{itemize}
\end{itemize}





\section*{Result}

\begin{enumerate}
    \item Notre Dame
    \begin{itemize}
        \item Matches:1275
        \item Accuracy on 50 most confident: 92\%
        \item Accuracy on 100 most confident: 73\%
        \item Accuracy on all matches: 16\%
    \end{itemize}
    
\begin{figure}[h]
    \centering
    \includegraphics[width=5cm]{nd1.png}
    \includegraphics[width=5cm]{nd2.png}
    \includegraphics[width=5cm]{nd3.png}
    \caption{My results for Notre Dame}
    \label{fig:result1}
\end{figure}
    
    \item Mount Rushmore
    \begin{itemize}
        \item Matches: 2022
        \item Accuracy on 50 most confident: 78\%
        \item Accuracy on 100 most confident: 71\%
        \item Accuracy on all matches: 8\%
    \end{itemize}
\end{enumerate}

\begin{figure}[h]
    \centering
    \includegraphics[width=5cm]{mt1.png}
    \includegraphics[width=5cm]{mt2.png}
    \includegraphics[width=5cm]{mt3.png}
    \caption{My results for Mount Rushmore}
    \label{fig:result1}
\end{figure}

\end{document}
